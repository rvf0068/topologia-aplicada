% Created 2021-01-18 Mon 20:05
% Intended LaTeX compiler: pdflatex
\documentclass[spanish, presentation]{beamer}
\usepackage[utf8]{inputenc}
\usepackage[T1]{fontenc}
\usepackage{graphicx}
\usepackage{grffile}
\usepackage{longtable}
\usepackage{wrapfig}
\usepackage{rotating}
\usepackage[normalem]{ulem}
\usepackage{amsmath}
\usepackage{textcomp}
\usepackage{amssymb}
\usepackage{capt-of}
\usepackage{hyperref}
\usepackage{lxfonts}
\usepackage[spanish, mexico, es-noshorthands, english]{babel}
% remove space between margin and lists
\usepackage{enumitem}
\setitemize{label=\usebeamerfont*{itemize item}%
\usebeamercolor[fg]{itemize item}
\usebeamertemplate{itemize item}}
\setlist{leftmargin=*,labelindent=0cm}
\setenumerate[1]{%
label=\protect\usebeamerfont{enumerate item}%
\protect\usebeamercolor[fg]{enumerate item}%
\insertenumlabel.}
\usepackage{listings}
\lstset{
literate={í}{{\'\i}}1
{á}{{\'a}}1
{é}{{\'e}}1
{ó}{{\'o}}1
{ú}{{\'u}}1
}
\lstalias{ipython}{python}
\usepackage{tikz}
\usepackage{tkz-berge}
\usetikzlibrary{graphs}
\usetikzlibrary{graphs.standard}
\usetheme[noframetitlerule]{Verona}
\author{Rafael Villarroel}
\date{2021-01-18 15:00 -0500}
\title{Inicio}
\beamerdefaultoverlayspecification{<+->}
\hypersetup{
 pdfauthor={Rafael Villarroel},
 pdftitle={Inicio},
 pdfkeywords={},
 pdfsubject={},
 pdfcreator={Emacs 27.1 (Org mode 9.4.4)}, 
 pdflang={English}}
\begin{document}

\maketitle
\languagepath{spanish}

En este curso vamos a estudiar temas de \alert{topología algebraica} que recientemente han encontrado aplicaciones. Las dos primeras partes son bastante teóricas, y en la última parte veremos varias aplicaciones interesantes.

Las estructuras central del curso son los \alert{complejos simpliciales}. Desde el punto de vista de las aplicaciones, podemos pensar a los complejos simpliciales como una generalización de las gráficas, en donde en lugar de modelar interacciones entre parejas de vértices, vamos a considerar el caso en que las interacciones se pueden dar entre cualesquier cantidad de vértices. A cada complejo simplicial se le puede asociar un espacio topológico. Y la idea es que los ``hoyos'' que se presentan en el espacio nos dan información sobre la situación modelada por el complejo simplicial.

La manera de detectar los ``hoyos'' es por medio de una construcción llamada ``homología''. Para definir la homología vamos a tener que repasar varios teoremas de Álgebra Lineal y examinar con cierto detalle las propiedades de grupos abelianos. Esto también nos dará ocasión de usar herramientas computacionales.
\end{document}