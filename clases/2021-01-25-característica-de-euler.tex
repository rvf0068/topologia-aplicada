% Created 2021-01-28 Thu 16:49
% Intended LaTeX compiler: pdflatex
\documentclass[spanish, presentation]{beamer}
\usepackage[utf8]{inputenc}
\usepackage[T1]{fontenc}
\usepackage{graphicx}
\usepackage{grffile}
\usepackage{longtable}
\usepackage{wrapfig}
\usepackage{rotating}
\usepackage[normalem]{ulem}
\usepackage{amsmath}
\usepackage{textcomp}
\usepackage{amssymb}
\usepackage{capt-of}
\usepackage{hyperref}
\usepackage{lxfonts}
\usepackage[spanish, mexico, es-noshorthands, english]{babel}
% remove space between margin and lists
\usepackage{enumitem}
\setitemize{label=\usebeamerfont*{itemize item}%
\usebeamercolor[fg]{itemize item}
\usebeamertemplate{itemize item}}
\setlist{leftmargin=*,labelindent=0cm}
\setenumerate[1]{%
label=\protect\usebeamerfont{enumerate item}%
\protect\usebeamercolor[fg]{enumerate item}%
\insertenumlabel.}
\usepackage{listings}
\lstset{
literate={í}{{\'\i}}1
{á}{{\'a}}1
{é}{{\'e}}1
{ó}{{\'o}}1
{ú}{{\'u}}1
}
\lstalias{ipython}{python}
\usepackage{tikz}
\usepackage{tkz-berge}
\usetikzlibrary{graphs}
\usetikzlibrary{graphs.standard}
\usetheme[noframetitlerule]{Verona}
\author{Rafael Villarroel}
\date{2021-01-25 15:00 -0500}
\title{Característica de Euler}
\beamerdefaultoverlayspecification{<+->}
\hypersetup{
 pdfauthor={Rafael Villarroel},
 pdftitle={Característica de Euler},
 pdfkeywords={},
 pdfsubject={},
 pdfcreator={Emacs 27.1 (Org mode 9.4.4)}, 
 pdflang={English}}
\begin{document}

\maketitle
\languagepath{spanish}

Sea \(\Delta\) un complejo simplicial de dimensión \(d\). Sea \(f_{i}(\Delta)\) igual a la cantidad de simplejos en \(\Delta\) de dimensión \(i\) para \(i=-1,0,1,\ldots,d\). El \alert{f-vector} de \(\Delta\) está definido como \(f(\Delta)=(f_{-1},f_{0},f_{1},\ldots,f_{d})\).

La \alert{característica (reducida) de Euler} \(\tilde{\chi}(\Delta)\) de \(\Delta\) se define como \(\tilde{\chi}(\Delta)=\sum_{i=-1}^{d}(-1)^{i}f_{i}(\Delta)\). (En general, durante el curso, toda característica de Euler será reducida)

Por ejemplo, si \(\Delta\) es el complejo simplicial con caras maximales \(12,13,23\), entonces \(\Delta\) tiene dimensión \(d=2\), su f-vector es \(f(\Delta)=(1,3,3)\), y su característica de Euler es \(\tilde{\chi}(\Delta)=-1+3-3=-1\).

Si \(\Delta=\mathcal{P}(\{1,2,\ldots,n\})\),  entonces \(\Delta\) tiene dimensión \(d=n-1\),  \(f_{i}(\Delta)=\binom{n}{i+1}\), y su característica de Euler es \(\tilde{\chi}(\Delta)=\sum_{i=-1}^{n-1}(-1)^{i}\binom{n}{i+1}=0\).

Similarmente, vimos varios ejemplos de triangulaciones de un polígono en \(\mathbb{R}^{2}\), y todas tuvieron característica de Euler igual a 0. (Es decir, observamos que si tenemos una triangulación del espacio \(D^{2}=\{x\in \mathbb{R}^{2}\mid |x|\leq 1\}\), su característica de Euler es 0).

También vimos triangulaciones de la esfera, como el octaedro y el icosaedro, y algunas triangulaciones no regulares, y todas ellas tuvieron característica de Euler igual a 1. Concluimos que el valor de la característica de Euler depende más de la \emph{forma}
 que de la \emph{métrica.}
\end{document}