% Created 2021-01-28 Thu 16:50
% Intended LaTeX compiler: pdflatex
\documentclass[spanish, presentation]{beamer}
\usepackage[utf8]{inputenc}
\usepackage[T1]{fontenc}
\usepackage{graphicx}
\usepackage{grffile}
\usepackage{longtable}
\usepackage{wrapfig}
\usepackage{rotating}
\usepackage[normalem]{ulem}
\usepackage{amsmath}
\usepackage{textcomp}
\usepackage{amssymb}
\usepackage{capt-of}
\usepackage{hyperref}
\usepackage{lxfonts}
\usepackage[spanish, mexico, es-noshorthands, english]{babel}
% remove space between margin and lists
\usepackage{enumitem}
\setitemize{label=\usebeamerfont*{itemize item}%
\usebeamercolor[fg]{itemize item}
\usebeamertemplate{itemize item}}
\setlist{leftmargin=*,labelindent=0cm}
\setenumerate[1]{%
label=\protect\usebeamerfont{enumerate item}%
\protect\usebeamercolor[fg]{enumerate item}%
\insertenumlabel.}
\usepackage{listings}
\lstset{
literate={í}{{\'\i}}1
{á}{{\'a}}1
{é}{{\'e}}1
{ó}{{\'o}}1
{ú}{{\'u}}1
}
\lstalias{ipython}{python}
\usepackage{tikz}
\usepackage{tkz-berge}
\usetikzlibrary{graphs}
\usetikzlibrary{graphs.standard}
\usetheme[noframetitlerule]{Verona}
\author{Rafael Villarroel}
\date{2021-01-28 15:00 -0500}
\title{Creación de complejos simpliciales}
\beamerdefaultoverlayspecification{<+->}
\hypersetup{
 pdfauthor={Rafael Villarroel},
 pdftitle={Creación de complejos simpliciales},
 pdfkeywords={},
 pdfsubject={},
 pdfcreator={Emacs 27.1 (Org mode 9.4.4)}, 
 pdflang={English}}
\begin{document}

\maketitle
\languagepath{spanish}

\section{Complejos de gráficas}
\label{sec:org5b3af85}

\begin{frame}[label={sec:org201fa53}]{El complejo de completas de una gráfica}
Sea \(G\) una gráfica (simple, finita). Una \alert{completa} de \(G\) es \(C\subseteq V(G)\) tal que si \(x_{1},x_{2}\in C\), entonces \(x_{1}\sim x_{2}\). Observemos que si \(C_{1}\) es completa de \(G\) y \(C_{2}\subseteq C_{1}\), entonces \(C_{2}\) es completa.

El complejo \(\Delta(G)\) se define como el complejo simplicial sobre \(V(G)\) cuyos simplejos son las completas de \(G\).

Si tenemos un complejo simplicial \(\Delta\) y existe una gráfica \(G\) tal que \(\Delta=\Delta(G)\), decimos que \(\Delta\) es un \alert{complejo simplicial de completas}. (En inglés, \(\Delta\) se llama \emph{flag complex} o \emph{clique complex}).

\alert{Ejemplos}. Sea \(G\) la gráfica donde el conjunto de vértices es \(V(G)=\{a,b,c,d,e,f\}\) , y el conjunto de aristas es: \(E(G)=\{ab,ac,ad,ae,af,bf,cf,de,ef\}\). Entonces se tiene que \(\mathcal{F}(\Delta(G))=\{abf,acf,ade,aef\}\).

\alert{Tarea}. Muestra que existe un complejo simplicial \(\Delta\) tal que no existe gráfica \(G\) con \(\Delta(G)=\Delta\).
\end{frame}

\begin{frame}[label={sec:orgfcb2b74}]{El complejo orientado de una digráfica}
Sea \(D\) una gráfica dirigida (cada arista tiene exactamente una dirección).  Vamos a formar un complejo simplicial \(\Delta^{\to}(D)\) sobre \(V(D)\), donde \(\sigma\subseteq V(D)\) es un simplejo si la subgráfica dirigida de \(D\) inducida por \(\sigma\) es completa y tiene un sumidero y una fuente.

\alert{Tarea.} Muestra que de verdad la construcción anterior define un complejo simplicial.
\end{frame}
\end{document}