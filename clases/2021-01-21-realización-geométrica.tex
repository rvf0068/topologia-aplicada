% Created 2021-02-01 Mon 21:17
% Intended LaTeX compiler: pdflatex
\documentclass[spanish, presentation]{beamer}
\usepackage[utf8]{inputenc}
\usepackage[T1]{fontenc}
\usepackage{graphicx}
\usepackage{grffile}
\usepackage{longtable}
\usepackage{wrapfig}
\usepackage{rotating}
\usepackage[normalem]{ulem}
\usepackage{amsmath}
\usepackage{textcomp}
\usepackage{amssymb}
\usepackage{capt-of}
\usepackage{hyperref}
\usepackage{lxfonts}
\usepackage[spanish, mexico, es-noshorthands, english]{babel}
% remove space between margin and lists
\usepackage{enumitem}
\setitemize{label=\usebeamerfont*{itemize item}%
\usebeamercolor[fg]{itemize item}
\usebeamertemplate{itemize item}}
\setlist{leftmargin=*,labelindent=0cm}
\setenumerate[1]{%
label=\protect\usebeamerfont{enumerate item}%
\protect\usebeamercolor[fg]{enumerate item}%
\insertenumlabel.}
\usepackage{listings}
\lstset{
literate={í}{{\'\i}}1
{á}{{\'a}}1
{é}{{\'e}}1
{ó}{{\'o}}1
{ú}{{\'u}}1
}
\lstalias{ipython}{python}
\usepackage{tikz}
\usepackage{tkz-berge}
\usetikzlibrary{graphs}
\usetikzlibrary{graphs.standard}
\usetheme[noframetitlerule]{Verona}
\author{Rafael Villarroel}
\date{2021-01-21 15:00 -0500}
\title{Realización geométrica}
\beamerdefaultoverlayspecification{<+->}
\hypersetup{
 pdfauthor={Rafael Villarroel},
 pdftitle={Realización geométrica},
 pdfkeywords={},
 pdfsubject={},
 pdfcreator={Emacs 27.1 (Org mode 9.4.4)}, 
 pdflang={English}}
\begin{document}

\maketitle
\languagepath{spanish}

\section{Idea}
\label{sec:org6f4050b}

\begin{frame}[label={sec:org1d6b0b0}]{Idea general}
A cada complejo simplicial \(\Delta\) le queremos asociar un espacio topológico que denotaremos con \(|\Delta|\), que se llama su \alert{realización geométrica}.

Vamos a considerar un número \(n\) suficientemente grande, y entonces \(|\Delta|\) será un subespacio de \(\mathbb{R}^{n}\).

Por cada simplejo de dimensión 0, ponemos un punto en \(\mathbb{R}^{n}\). Por cada simplejo de dimensión \(1\), ponemos un segmento de recta entre los puntos del simplejo. Por cada simplejo de dimensión 2, ponemos un triángulo entre sus vértices, etc. 
\end{frame}

\begin{frame}[label={sec:orgf7bb8fe}]{Ejemplos}
\begin{itemize}
\item Consideremos el complejo simplicial \(\Delta_{1}\) con facetas \(\{\{1,2,3\},\{2,4\}\}\).
\item \(\Delta_{2}\) con caras maximales \(\{12,23,34,145\}\).
\item \(\Delta_{3}\) con caras maximales \(\{012,123,03\}\).
\item \(\Delta_{4}\) con caras maximales \(\{145,246,356,12,23,13\}\).
\item \(\Delta_{5}\) con caras maximales \(\{12345\}\) (4-simplejo).
\item \alert{Tarea:} \(\Delta_{6}\) con caras maximales \(\{124,126,134,135,156,235,245,236,346,456\}\).
\end{itemize}
\end{frame}

\section{Definición formal}
\label{sec:org7c1d3f6}

\begin{frame}[label={sec:org8224da6}]{Independencia afín}
Decimos que el conjunto \(\{v_{0},v_{1},\ldots,v_{d}\}\subseteq \mathbb{R}^{n}\) es \alert{afínmente independiente}
si \(\{v_{1}-v_{0},v_{2}-v_{0},\ldots,v_{d}-v_{0}\}\subseteq \mathbb{R}^{n}\) es \alert{linealmente independiente}. Por convención, todo conjunto con un solo punto es afínmente independiente.

Cuando un conjunto de puntos en \(\mathbb{R}^{n}\) es afínmente independiente, también se dice que los puntos están en \alert{posición general}.

\alert{Tarea}
Demuestra que el conjunto \(\{v_{0},v_{1},\ldots,v_{d}\}\) es afínmente independiente si y solo si \(\sum_{i=0}^{d}t_{i}v_{i}=0\) y \(\sum_{i=0}^{d}t_{i}=0\) implican que \(t_{0}=t_{1}=\cdots=t_{d}=0\). Por lo tanto, la propiedad de que un conjunto sea afínmente independiente no depende del punto escogido como \(v_{0}\).

\alert{Ejemplo}
En \(\mathbb{R}^{3}\), el conjunto \(\{(0,0,0),(1,0,0),(0,1,0),(0,0,1)\}\) es afínmente independiente.

\alert{Tarea}
Demuestra que si un conjunto no es afínmente independiente, pasa uno de:
\begin{itemize}
\item hay tres puntos colineales
\item hay cuatro puntos coplanares
\item etc.
\end{itemize}
\end{frame}

\begin{frame}[label={sec:org3eb3ac2}]{Simplejo geométrico}
Consideremos el conjunto \(A=\{v_{0},v_{1},\ldots,v_{d}\}\subseteq \mathbb{R}^{n}\) que sea afínmente independiente. El \alert{simplejo geométrico generado por \(A\)} es el subespacio: \(|A|=\{\sum_{i=0}^{d}t_{i}v_{i}\mid t_{i}\geq 0, \sum_{i=0}^{d}t_{i}=1\}\subseteq \mathbb{R}^{n}\).

Por ejemplo, el simplejo geométrico generado por un punto es el mismo punto. Si \(A=\{v_{0},v_{1}\}\), entonces \(|A|\) es el segmento de recta de \(v_{0}\) a \(v_{1}\). Si \(A=\{v_{0},v_{1},v_{2}\}\), entonces \(|A|\) es el triángulo con vértices \(v_{0},v_{1},v_{2}\), etc.

Todo elemento de la forma \(\sum_{i=0}^{d}t_{i}v_{i}\)  con \(t_{i}\geq0\)  y \(\sum_{i=0}^{d}t_{i}=1\), se llama una \alert{combinación convexa} de \(v_{0},v_{1},\ldots,v_{d}\).

\alert{Tarea}
Demuestra que si \(\alpha\in|A|\), los números \(t_{i}\) tales que \(\alpha=\sum_{i=0}^{d}t_{i}v_{i}\) están unívocamente determinados.
\end{frame}

\begin{frame}[label={sec:org9ca7045}]{Realización geométrica}
\begin{block}{Vértices de un complejo simplicial}
Si \(\Delta\) es un complejo simplicial en \(X\), denotamos con \(\Delta_{0}=\{x\in X\mid \{x\}\in \Delta\}\). Los elementos de \(\Delta_{0}\) se llaman \alert{vértices} de \(\Delta\).
\end{block}

\begin{block}{Realización de un simplejo}
Sean \(\Delta\) un complejo simplicial, \(\sigma\in\Delta\), y sea \(\phi\colon\Delta_{0}\to \mathbb{R}^{n}\) una función tal que \(\phi(\sigma)=\{\phi(x)\mid x\in\sigma\}\) es afínmente independiente. Definimos \(|\sigma|_{\phi}\) como el simplejo geométrico generado por \(A=\phi(\sigma)\).
\end{block}

\begin{block}{Encaje afín}
Sea \(\Delta\) un complejo simplicial. Decimos que  \(\phi\colon \Delta_{0}\to\mathbb{R}^{n}\) es un \alert{encaje afín} si:
\begin{itemize}
\item \(\phi\) es inyectiva,
\item \(\phi(\sigma)\) es afínmente independiente para todo \(\sigma\in\Delta\),
\item Para cada \(\sigma,\tau\in\Delta\) tales que \(|\sigma|_{\phi}\cap|\tau|_{\phi}\ne\emptyset\), se tiene que existe \(\rho\in\Delta\) tal que \(|\rho|_{\phi}=|\sigma|_{\phi}\cap|\tau|_{\phi}\).
\end{itemize}
\end{block}

\begin{block}{Ejemplo.}
Si \(\phi\) es inyectiva y \(\phi(\Delta_{0})\) es afínmente independiente, entonces \(\phi\) es un encaje afín.
\end{block}

\begin{block}{Realización respecto a un encaje}
Sea \(\Delta\) un complejo simplicial. Sea \(\phi\colon \Delta_{0}\to \mathbb{R}^{n}\) un encaje afín. Entonces se define
\begin{equation}
\label{eq:1}
|\Delta|_{\phi}=\cup_{\sigma\in\Delta}|\sigma|_{\phi}.
\end{equation}
\end{block}
\end{frame}
\end{document}